\SSbreak\\       
\emph{Source: OMO, Fall 2017 P28}\\
\emph{Proposer: .19\#9839 (434767660182405131)}\\
\emph{Problem ID: 62}\\
\emph{Date: 2020-11-30}\\
\SSbreak

\SSpsetQ{
    Define a triangle \(ABC\), with sides \(AB:AC:BC=7:9:10\). Further, for the circumcircle of \(ABC\), \(\omega\), let the circumcenter be \(O\), and the circumradius to be \(R\). The tangets to \(\omega\) at points \(B\) and \(C\) meet at \(X\), and a variable line \(l\) passes through \(O\). Define \(A_1\) to be the projection of \(X\) onto \(l\), and \(A_2\) to be the reflection of \(A_1\) over \(O\). Suppose that there exists two points \(Y,\ Z\) on \(l\) such that \(\angle YAB+\angle YBC+\angle YCA=\angle ZAB+\angle ZCA=90^{\circ}\), where all angles are directed, furthermore that \(O\) lies inside segment \(YZ\) with \(OY\*OZ=R^2\). Then there are several possible values for the sine of the angle at which the angle bisector of \(\angle AA_2O\) meets \(BC\). If the product of these values can be expressed in the form \(\frac{a\sqrt{b}}{c}\) for positive integers \(a,b,c\), with \(b\) squarefree and \(a,c\) coprime, determine \(a+b+c\).
}\bigskip

\begin{solution}\hfil\medskip

    \href{https://internetolympiad.org/archive/OMOFall17/OMOFall17Solns.pdf}{OMO Fall 2017 solutions (P28)} 

\end{solution}