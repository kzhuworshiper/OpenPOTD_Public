\SSbreak\\
\emph{Source: United Kingdom Senior Mathematical Challenge, 2015 Q23}\\
\emph{Proposer: Unknown}\\
\emph{Problem ID: 27}\\
\emph{Date: 2020-11-04}\\
\SSbreak

\SSpsetQ{
    Given four different non-zero digits, it is possible to form 24 different numbers containing each of these four digits. What is the largest prime factor of the sum of the 24 numbers?
}\bigskip

\begin{solution}\hfil\medskip 

    Say the digits are $a$, $b$, $c$, and $d$. For each digit, it appears in the units place 6 times, the tens place 6 times, the hundreds place 6 times, and the thousands place 6 times. \\

    Hence the sum of the 24 numbers is $6666(a+b+c+d) = 2 \cdot 3 \cdot 11 \cdot 101 (a+b+c+d)$. Since $a+b+c+d < 40$, it cannot have any prime factors larger than 101. \\

    The largest prime factor of the sum of the 24 numbers is thus $\boxed{101}$. 
\end{solution}\bigskip