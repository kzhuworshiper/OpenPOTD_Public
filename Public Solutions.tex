\documentclass[titlepage=true]{scrartcl}

\usepackage{preamble}

\def\Ptony{Tony Wang\#1729 (541318134699786272)}
\def\Pbrain{brainysmurfs\#2860 (281300961312374785)}
\def\Pss{.19\#9839 (434767660182405131)}
\def\Ppi{Charge\#3766(481250375786037258)}
\def\Pbfan{bfan05\#5219 (692851547062665317)}
\def\Paiya{AiYa\#2278 (675537018868072458)}
\def\Plook{Look At Me\#8623 (234848703368658944)}

\begin{document}
\SSfp

This is a temporary document to let people who have alternative solutions contribute to the end of season write up.
Solutions have already been written up for the entire season, and where possible I have tried to include the officially provided solution, adapted them if the problem statement is changed, and filled in the gaps as best I could.
In many circumstances problems for this season have not come with official write-ups and thus have required me to provide my own - in which case I apologise for any mistakes (or fakesolves!) in advance.
If any mistakes are found, feel free to DM me \(.19\#9839\) or submit a push.
Similarly, if you would like to contribute an alternate solution to whatever is in this document, again, feel free to submit a push or just DM me a solution and I can type it up.
\medskip
 
At the end of each day I'll add the question and solution to the document - the end of each day naturally being such that the question and solution being added is no longer an active one.\bigskip

Thank you to:\medskip

\Paiya\medskip

For contributing to solutions! 

\newpage
    
\section{A Sequence of 5's}

    \SSbreak\\
    \emph{Source: United Kingdom - Maclaurin, 2015 M1}\\
    \emph{Proposer: .19\#9839 (434767660182405131)}\\
    \emph{Problem ID: 47}\\
    \emph{Date: 2020-11-16}\\
    \SSbreak
        
    \SSpsetQ{
        Consider the sequence \(5,\ 55,\ 555,\ 5555,\ldots\)\medskip

        How many digits does the smallest number in the sequence have which is divisible by 495?
    }\bigskip

    \begin{solution}\hfil\medskip

        We require the term to be divisible by \(5\cdot9\cdot 11\).
		Hence we need only consider the sequence \(1,\ 11,\ 111\,\ldots\) with respect to \(9\cdot11\).
		Clearly for odd numbered terms in the sequence, 11 does not divide into it, by the well-known divisibility rule for 11.
		Therefore, we require an even numbered term in the sequence, which is divisible by 9.
		We know 9 divides a number iff its digital sum is also divisible by 9.
		Hence, the smallest such will be the 18th term in the sequence, which will naturally have \fbox{18} digits. 
    \end{solution}\bigskip


    \begin{solution}[Write up by \Paiya]\hfil\medskip
        
        Each of these numbers can be written as \(5\cdot1\ldots1,\)whee there are \(n\) total ones.
		This can be rewritten as \(5\cdot(10^{n-1}+10^{n-2}+\cdots+10^0)=\frac{5}{9}(10^n-1)\).
		Note that \(495=9\cdot11\cdot5\) so we want \(9|\frac{10^n-1}{9}\) and \(10^n\equiv1\mod{11}\).
		From the congruence \(\mod{11}\) we see that \(n\) must be even.
		Note that \(10^n-1=9\ldots9\), where there are \(n\) total nines; if \(n\) is a multiple of 9 then \(\frac{10^n-1}{9}=1\ldots1\) where there are \(n\) total ones; this is a multiple of 9.
		Since \(n\) must be even, the smallest such \(n\) is \fbox{18}.
    \end{solution}

\newpage

\section{Brainy's Happy Set}

    \SSbreak\\
    \emph{Source: British Mathematical Olympiad - Round 1, 2010/2011 P1}\\
    \emph{Proposer: .19\#9839 (434767660182405131)}\\
    \emph{Problem ID: 40}\\
    \emph{Date: 2020-11-17}\\
    \SSbreak
        
    \SSpsetQ{
        Brainy has a set of integers, from 1 to \(n\), which he likes to play with.
		Tony Wang, upon seeing the happiness that this set of integers brings Brainy, decides to steal one of the numbers in it.
		Suppose the average number of the remaining elements in the set is \(\frac{163}{4}\).
		What is the sum of the elements in Brainy's set multiplied by the element that Tony stole?
                
        \begin{center}
            \emph{(A four-function calculator may be used)}
        \end{center}
    }\bigskip
   
    \begin{solution}\hfil\medskip 
   
        We can set up the problem statement as 
        \begin{equation*}
            \frac{\frac{n}{2}(n+1)-x}{n-1}=\frac{163}{4}
        \end{equation*}
                
         Where \(x\) is the number Tony has stolen.
	 This simplifies to \(4x=2n^2-161n+163\).
	 Since \(x\) must be a number within the set \(\{1,2,\ldots,n-1,n\}\), we have that \(1\leq x\leq n\Rightarrow 4\leq 2n^2-161n+163\leq4n\).
	 By considering the lower bound, we get $(2n-159)(n-1)\geq 0$.
	 This means that \(n\leq 1\Rightarrow n=1\), or \(n\geq \frac{159}{2}\Rightarrow n\geq 80\).
	 By similar methodology when considering the upper bound, we get \(1\leq n\leq 81\).
	 Thus \(n\in\{1,80,81\}\).
	 Clearly, \(n\ne 1\), so either \(n=80\) or \(n=81\).
	 Notice that if \(n\) is even, then for \(4x=2n^2-161n+163\) the parity of he RHS is Odd, while the LHS is even, thus a contradiction occurs.
	 This means that \(n=81\) and so \(x=61\). Thus the answer is \(\frac{81(82)}{2}\cdot61=\fbox{202581}\).
    \end{solution}\bigskip

%    This solution's pretty much redundant
%    \begin{solution}[Write up by \Paiya]\hfil\medskip
%        
%        Suppose Tony stole the integer \(k\).
%		Then, the sum of Brainy's remaining \(n-1\) elements is \(\frac{n(n+1)}{2}-k\) for an average of 
%        
%        \begin{equation*}
%            \frac{\frac{n(n+1)}{2}-k}{n-1}=\frac{163}{4}
%        \end{equation*}
%
%        Rearrange and solve for \(k\) to get \(4k=2n^2=161n+163\).
%		Taking this \(\mod{4}\), we get \(2n^2-n-1\equiv0\mod{4}\) so \(n\equiv1\mod{4}\).
%		Since \(k\) is natural, we have \(2n^2-161n+163>0\iff n>\frac{161+\sqrt{161^2-8\cdot163}}{4}=79.4\) and \(n=81\) gives us \(k=61\). 
%		So the answer is \(\fbox{202581}\).
%
%        \begin{remark}
%            81 is the unique solution. We must have \(k\geq n\) so \(2n^2-161n+163\geq 4n\iff n\geq 81.5\)
%        \end{remark}
%    \end{solution}\bigskip

\newpage

\section{MODSbot's Escape!}

    \SSbreak\\
    \emph{Source: Mathematics Admissions Test, 2012 Q5}\\
    \emph{Proposer: .19\#9839 (434767660182405131)}\\
    \emph{Problem ID: 48}\\
    \emph{Date: 2020-11-18}\\
    \SSbreak

    \SSpsetQ{
        In his evil mechatronics laboratory, Brainy has built a physical manifestation of MODSbot.
		MODSbot's movement is defined by three inputs: \textbf{F} to move forward a unit distance, \textbf{L} to turn left \(90^{\circ}\), and \textbf{R} to turn right \(90^{\circ}\).

        We define a program to be a sequence of commands.
		The program \(P_{n+1}\) (for \(n\geq 0\)) involves performing \(P_n\), turning left, performing \(P_n\) again, then turning right:\medskip

        \[P_{n+1}=P_n\textbf{L}P_n\textbf{R},\ P_0=\textbf{F}\]\medskip

        Unbeknownst to Brainy, MODSbot, though limited in movement, is sentient and realises Brainy is just a small asian Frankenstein, whose intentions for them were nefarious and non-consensual.
		As a result, after Brainy goes home for the day, MODSbot makes its escape from Brainy's laboratory.\medskip

        Let \((x_n,y_n)\) be the position of the robot after performing the program \(P_n\), so \((x_0,y_0)=(1,0)\) and \((x_1,y_1)=(1,1)\), etc.\medskip
   
        How far away from the place Brainy left it does MODSbot make it after performing\(P_{24}\)?
    }\bigskip


    \begin{solution}\hfil\medskip
            
        Note first that after each iteration of $P_n$ MODSbot faces in the positive $x$ direction, as each $P_n$ contains as many \textbf{L}s as it does \textbf{R}s.
		Now, assuming MODSbot is at $(x_n,y_n)$ after having performed $P_n$, we see the next iteration of $P$ puts MODSbot at $(x_n-y_n,x_n+y_n)$.
		Note then that:
            
            \begin{align*}
                (x_{n+2},y_{n+2}) &= (x_{n+1}-y_{n+1},x_{n+1}+y_{n+1}) = (-2y_n,2x_n)\\
                (x_{n+4},y_{n+4}) &= (-2y_{n+2},2x_{n+2})  = (-4x_n,-4y_n)\\
                (x_{n+8},y_{n+8}) &= (-4x_{n+4},4y_{n+4})  = (16x_n,16y_n)
            \end{align*}
        
        Thus, we see that $(x_{8k},y_{8k}) = (16^k,0)$, and therefore that $\abs{P_{24}} = \fbox{4096}$
    \end{solution}\bigskip

    \begin{solution}[Write up by \Paiya]\hfil\medskip
        
        Observe that each program has the same amount of left and right turns, so MODSbot will always be facing the positive \(x\)-direction after each program.
		This means that \(\textbf{L}P_n\) is just the program \(P_n\) performed at a 90-degree counterclock-wise rotation.
		For instance \(P_1\) moves MODSbot right 1 and up 1, so \(\textbf{L}P_1\) moves MODSbot up 1 and left 1 (right gets rotated 90 counterclockwise to up and up to left).
		This motivates us to work in the complex plane; let \(P_n\) be the complex-number representing MODSBOT's displacement after following \(P_n\).
		Then \(\textbf{L}P_n=iP_n\), so \(P_{n+1}=P_n+\textbf{L}P_n=(1+i)P_n=\sqrt{2}e^{\frac{\pi i}{4}}P_n\).
		WIth \(P_0-1\) we get \(P_n=2^{\frac{n}{2}}e^\frac{\pi i n}{4}\). 
		So \(|P_{24}=\fbox{4096}\)
    \end{solution}\bigskip
\newpage

\section{Sides of a Polygon}

    \SSbreak\\
    \emph{Source: Folklaw\footnote{This has appeared on a Polish MO, British MO 1966 P4, an 2018 NZ IMO handout, a WOOT handout, to name a few...}}\\
    \emph{Proposer: .19\#9839 (434767660182405131)}\\
    \emph{Problem ID: 53}\\
    \emph{Date: 2020-11-19}\\
    \SSbreak

    \SSpsetQ{
        Points \(A,\ B,\ C,\ D\) are the consecutive vertices of a regular polygon, and the following relation holds:

        \begin{equation*}
            \frac{1}{AB}=\frac{1}{AC}+\frac{1}{AD}
        \end{equation*}

        How many sides does this polygon have?
    }\bigskip

    \begin{solution}\hfil\medskip

        Drawing out the general shape of an \(n\)-gon - as seen in the figure - and letting \(OA=OB=OC=OD\cdots=1\), and \(\angle MOA=x\). By the sine rule on \(\triangle AMO\), and noting \(AM=\frac{1}{2}AB\), we get \(\frac{1}{AB}=\frac{2}{\sin x}\). By a similar procedure, this time on \(\triangle ACO\), we see \(\frac{1}{AB}=\frac{2}{\sin 2x}\), and again on \(\triangle ADO\), we have \(\frac{1}{AD}=\frac{2}{\sin3x}\). Therefore we have the equality: 
        
        \begin{align*}
            \frac{1}{\sin x}&=\frac{1}{\sin2x}+\frac{1}{\sin3x}
        \end{align*}

        Simplifying this yields \(\sin x\sin\frac{x}{2}\sin\frac{7x}{2}=0\). However, note that \(x\ne\frac{k\pi}{2},\ \frac{k\pi}{3}\) for \(k\in\mathbb{Z}\), otherwise, we have the issue of dividing by 0. Hence it must be the case that \(\sin\frac{7x}{2}=0\Rightarrow x=\frac{\pi}{7}+\frac{2k\pi}{7}\). Clearly the \(n\)-gon is not a square, so trivially it must be the case that \(x=\frac{\pi}{7}\). Therefore, the polygon must have \fbox{7} sides.
        \begin{figure}[h!]
            \centering
            \scalebox{0.8}{\begin{tikzpicture}[line cap=round,line join=round,>=triangle 45,x=1.0cm,y=1.0cm]
            \clip(-5.5,-5.5) rectangle (5.5,5.5);
            \draw [shift={(0.,0.)},line width=0.4pt] (0,0) -- (-67.5:0.3710168360883199) arc (-67.5:-22.5:0.3710168360883199) -- cycle;
            \draw [shift={(0.,0.)},line width=0.4pt] (0,0) -- (-22.5:0.3710168360883199) arc (-22.5:22.5:0.3710168360883199) -- cycle;
            \draw [shift={(0.,0.)},line width=0.4pt] (0,0) -- (22.5:0.3710168360883199) arc (22.5:67.5:0.3710168360883199) -- cycle;
            \draw[line width=0.4pt] (3.0808989941467813,-3.0808989941467813) -- (2.895390576102621,-3.2664074121909414) -- (3.080898994146781,-3.4519158302351016) -- (3.266407412190941,-3.2664074121909414) -- cycle; 
            \draw [line width=0.4pt,] (1.9134171618254487,-4.619397662556435)-- (4.619397662556434,-1.9134171618254485);
            \draw [line width=0.4pt,dash pattern=on 5pt off 5pt] (4.619397662556434,-1.9134171618254485)-- (4.619397662556434,1.9134171618254485);
            \draw [line width=0.4pt,dash pattern=on 5pt off 5pt] (4.619397662556434,1.9134171618254485)-- (1.913417161825449,4.619397662556434);
            \draw [line width=0.4pt,dash pattern=on 5pt off 5pt] (1.913417161825449,4.619397662556434)-- (-1.9134171618254483,4.619397662556434);
            \draw [line width=0.4pt] (1.9134171618254487,-4.619397662556435)-- (4.619397662556434,1.9134171618254485);
            \draw [line width=0.4pt] (1.9134171618254487,-4.619397662556435)-- (1.913417161825449,4.619397662556434);
            \draw [line width=0.4pt] (0.,0.)-- (1.9134171618254487,-4.619397662556435);
            \draw [line width=0.4pt] (0.,0.)-- (4.619397662556434,-1.9134171618254485);
            \draw [line width=0.4pt] (0.,0.)-- (4.619397662556434,1.9134171618254485);
            \draw [line width=0.4pt] (0.,0.)-- (1.913417161825449,4.619397662556434);
            \draw [line width=0.4pt] (0.,0.)-- (3.266407412190941,-3.2664074121909414);
            \begin{scriptsize}
            \draw[color=black] (-0.25016257386958146,3.362996814193775E-4) node {$O$};
            \draw[color=black] (4.98117481497573,2.078030581776008) node {$C$};
            \draw[color=black] (1.8275317082250102,-4.934187620293228) node {$A$};
            \draw[color=black] (5.111030707606642,-2.114459666022001) node {$B$};
            \draw[color=black] (2.050141809878002,4.916309377851651) node {$D$};
            \draw[color=black] (-1.9753908616802691,4.9719619032649) node {$E$};
            \draw[color=black] (3.4971074706224496,-3.394467750526703) node {$M$};
            \end{scriptsize}
            \end{tikzpicture}}
            \caption*{A regular $n$-gon}
        \end{figure}
    \end{solution}
    
    \begin{solution}[Write up by \Paiya]\hfil\medskip
    
      Let $d_k$ represent the diagonal from a point to the kth vertex adjacent to it.     For example, $d_1$ is a side of the polygon, $d_2$ is $\overline{AC}, d_3$ is $\overline{AD}$ and note that $d_k = d_{n - k}$ where n is the number of sides of the polygon. Reassign $D$ to be the vertex three vertices away from $A$ but on the opposite side of $B$ and $C;$ in other words, reflect $D$ over $\overline{OA}.$ Then, $AB = BC = d_1, AC = d_2, AD = d_3, BD = d_4,$ and $CD = d_5.$ By Ptolemy's Theorem, we get $$AB \cdot CD + BC \cdot AD = AC \cdot BD \iff d_1\left(d_5 + d_3\right) = d_2d_4.$$ Rearrange our given equation to get $$\dfrac{1}{d_1} = \dfrac{1}{d_2} + \dfrac{1}{d_3} \iff d_1\left(d_2 + d_3\right) = d_2d_3.$$ For both of these equations to be true, we can have $d_5 + d_3 = d_2 + d_3 \iff d_5 = d_2$ and $d_3 = d_4;$ this is true if n = \boxed{7}.
    \end{solution}

\end{document} 
