\documentclass[titlepage=true]{scrartcl}

\usepackage{preamble}

\def\Ptony{Tony Wang\#1729 (541318134699786272)}
\def\Pbrain{brainysmurfs\#2860 (281300961312374785)}
\def\Pss{.19\#9839 (434767660182405131)}
\def\Ppi{Charge\#3766(481250375786037258)}
\def\Pbfan{bfan05\#5219 (692851547062665317)}
\def\Paiya{AiYa\#2278 (675537018868072458)}
\def\Plook{Look At Me\#8623 (234848703368658944)}

\begin{document}
\SSfp

This is a temporary document to let people who have alternative solutions contribute to the end of season write up.
Solutions have already been written up for the entire season, and where possible I have tried to include the officially provided solution, adapted them if the problem statement is changed, and filled in the gaps as best I could.
In many circumstances problems for this season have not come with official write-ups and thus have required me to provide my own - in which case I apologise for any mistakes (or fakesolves!) in advance.
If any mistakes are found, feel free to DM me \(.19\#9839\) or submit a push.
Similarly, if you would like to contribute an alternate solution to whatever is in this document, again, feel free to submit a push or just DM me a solution and I can type it up.
\medskip
 
At the end of each day I'll add the question and solution to the document - the end of each day naturally being such that the question and solution being added is no longer an active one.\bigskip

Thank you to:\medskip

\Paiya\medskip

For contributing to solutions! 

\newpage
    
\section{A Sequence of 5's}

    \SSbreak\\
    \emph{Source: United Kingdom - Maclaurin, 2015 M1}\\
    \emph{Proposer: .19\#9839 (434767660182405131)}\\
    \emph{Problem ID: 47}\\
    \emph{Date: 2020-11-16}\\
    \SSbreak
        
    \SSpsetQ{
        Consider the sequence \(5,\ 55,\ 555,\ 5555,\ldots\)\medskip

        How many digits does the smallest number in the sequence have which is divisible by 495?
    }\bigskip

    \begin{solution}\hfil\medskip

        We require the term to be divisible by \(5\cdot9\cdot 11\).
		Hence we need only consider the sequence \(1,\ 11,\ 111\,\ldots\) with respect to \(9\cdot11\).
		Clearly for odd numbered terms in the sequence, 11 does not divide into it, by the well-known divisibility rule for 11.
		Therefore, we require an even numbered term in the sequence, which is divisible by 9.
		We know 9 divides a number iff its digital sum is also divisible by 9.
		Hence, the smallest such will be the 18th term in the sequence, which will naturally have \fbox{18} digits. 
    \end{solution}\bigskip


    \begin{solution}[Write up by \Paiya]\hfil\medskip
        
        Each of these numbers can be written as \(5\cdot1\ldots1,\)whee there are \(n\) total ones.
		This can be rewritten as \(5\cdot(10^{n-1}+10^{n-2}+\cdots+10^0)=\frac{5}{9}(10^n-1)\).
		Note that \(495=9\cdot11\cdot5\) so we want \(9|\frac{10^n-1}{9}\) and \(10^n\equiv1\mod{11}\).
		From the congruence \(\mod{11}\) we see that \(n\) must be even.
		Note that \(10^n-1=9\ldots9\), where there are \(n\) total nines; if \(n\) is a multiple of 9 then \(\frac{10^n-1}{9}=1\ldots1\) where there are \(n\) total ones; this is a multiple of 9.
		Since \(n\) must be even, the smallest such \(n\) is \fbox{18}.
    \end{solution}

\newpage

\section{Brainy's Happy Set}

    \SSbreak\\
    \emph{Source: British Mathematical Olympiad - Round 1, 2010/2011 P1}\\
    \emph{Proposer: .19\#9839 (434767660182405131)}\\
    \emph{Problem ID: 40}\\
    \emph{Date: 2020-11-17}\\
    \SSbreak
        
    \SSpsetQ{
        Brainy has a set of integers, from 1 to \(n\), which he likes to play with.
		Tony Wang, upon seeing the happiness that this set of integers brings Brainy, decides to steal one of the numbers in it.
		Suppose the average number of the remaining elements in the set is \(\frac{163}{4}\).
		What is the sum of the elements in Brainy's set multiplied by the element that Tony stole?
                
        \begin{center}
            \emph{(A four-function calculator may be used)}
        \end{center}
    }\bigskip
   
    \begin{solution}\hfil\medskip 
   
        We can set up the problem statement as 
        \begin{equation*}
            \frac{\frac{n}{2}(n+1)-x}{n-1}=\frac{163}{4}
        \end{equation*}
                
         Where \(x\) is the number Tony has stolen.
	 This simplifies to \(4x=2n^2-161n+163\).
	 Since \(x\) must be a number within the set \(\{1,2,\ldots,n-1,n\}\), we have that \(1\leq x\leq n\Rightarrow 4\leq 2n^2-161n+163\leq4n\).
	 By considering the lower bound, we get $(2n-159)(n-1)\geq 0$.
	 This means that \(n\leq 1\Rightarrow n=1\), or \(n\geq \frac{159}{2}\Rightarrow n\geq 80\).
	 By similar methodology when considering the upper bound, we get \(1\leq n\leq 81\).
	 Thus \(n\in\{1,80,81\}\).
	 Clearly, \(n\ne 1\), so either \(n=80\) or \(n=81\).
	 Notice that if \(n\) is even, then for \(4x=2n^2-161n+163\) the parity of he RHS is Odd, while the LHS is even, thus a contradiction occurs.
	 This means that \(n=81\) and so \(x=61\). Thus the answer is \(\frac{81(82)}{2}\cdot61=\fbox{202581}\).
    \end{solution}\bigskip

\newpage

\section{MODSbot's Escape!}

    \SSbreak\\
    \emph{Source: Mathematics Admissions Test, 2012 Q5}\\
    \emph{Proposer: .19\#9839 (434767660182405131)}\\
    \emph{Problem ID: 48}\\
    \emph{Date: 2020-11-18}\\
    \SSbreak

    \SSpsetQ{
        In his evil mechatronics laboratory, Brainy has built a physical manifestation of MODSbot.
		MODSbot's movement is defined by three inputs: \textbf{F} to move forward a unit distance, \textbf{L} to turn left \(90^{\circ}\), and \textbf{R} to turn right \(90^{\circ}\).

        We define a program to be a sequence of commands.
		The program \(P_{n+1}\) (for \(n\geq 0\)) involves performing \(P_n\), turning left, performing \(P_n\) again, then turning right:\medskip

        \[P_{n+1}=P_n\textbf{L}P_n\textbf{R},\ P_0=\textbf{F}\]\medskip

        Unbeknownst to Brainy, MODSbot, though limited in movement, is sentient and realises Brainy is just a small asian Frankenstein, whose intentions for them were nefarious and non-consensual.
		As a result, after Brainy goes home for the day, MODSbot makes its escape from Brainy's laboratory.\medskip

        Let \((x_n,y_n)\) be the position of the robot after performing the program \(P_n\), so \((x_0,y_0)=(1,0)\) and \((x_1,y_1)=(1,1)\), etc.\medskip
   
        How far away from the place Brainy left it does MODSbot make it after performing\(P_{24}\)?
    }\bigskip


    \begin{solution}\hfil\medskip
            
        Note first that after each iteration of $P_n$ MODSbot faces in the positive $x$ direction, as each $P_n$ contains as many \textbf{L}s as it does \textbf{R}s.
		Now, assuming MODSbot is at $(x_n,y_n)$ after having performed $P_n$, we see the next iteration of $P$ puts MODSbot at $(x_n-y_n,x_n+y_n)$.
		Note then that:
            
            \begin{align*}
                (x_{n+2},y_{n+2}) &= (x_{n+1}-y_{n+1},x_{n+1}+y_{n+1}) = (-2y_n,2x_n)\\
                (x_{n+4},y_{n+4}) &= (-2y_{n+2},2x_{n+2})  = (-4x_n,-4y_n)\\
                (x_{n+8},y_{n+8}) &= (-4x_{n+4},4y_{n+4})  = (16x_n,16y_n)
            \end{align*}
        
        Thus, we see that $(x_{8k},y_{8k}) = (16^k,0)$, and therefore that $\abs{P_{24}} = \fbox{4096}$
    \end{solution}\bigskip

    \begin{solution}[Write up by \Paiya]\hfil\medskip
        
        Observe that each program has the same amount of left and right turns, so MODSbot will always be facing the positive \(x\)-direction after each program.
		This means that \(\textbf{L}P_n\) is just the program \(P_n\) performed at a 90-degree counterclock-wise rotation.
		For instance \(P_1\) moves MODSbot right 1 and up 1, so \(\textbf{L}P_1\) moves MODSbot up 1 and left 1 (right gets rotated 90 counterclockwise to up and up to left).
		This motivates us to work in the complex plane; let \(P_n\) be the complex-number representing MODSBOT's displacement after following \(P_n\).
		Then \(\textbf{L}P_n=iP_n\), so \(P_{n+1}=P_n+\textbf{L}P_n=(1+i)P_n=\sqrt{2}e^{\frac{\pi i}{4}}P_n\).
		WIth \(P_0=1\) we get \(P_n=2^{\frac{n}{2}}e^\frac{\pi i n}{4}\). 
		So \(|P_{24}=\fbox{4096}\)
    \end{solution}\bigskip
\newpage

\section{Sides of a Polygon}

    \SSbreak\\
    \emph{Source: Folklaw\footnote{This has appeared on a Polish MO, British MO 1966 P4, an 2018 NZ IMO handout, a WOOT handout, to name a few...}}\\
    \emph{Proposer: .19\#9839 (434767660182405131)}\\
    \emph{Problem ID: 53}\\
    \emph{Date: 2020-11-19}\\
    \SSbreak

    \SSpsetQ{
        Points \(A,\ B,\ C,\ D\) are the consecutive vertices of a regular polygon, and the following relation holds:

        \begin{equation*}
            \frac{1}{AB}=\frac{1}{AC}+\frac{1}{AD}
        \end{equation*}

        How many sides does this polygon have?
    }\bigskip

    \begin{solution}\hfil\medskip

        Drawing out the general shape of an \(n\)-gon - as seen in the figure - and letting \(OA=OB=OC=OD\cdots=1\), and \(\angle MOA=x\). By the sine rule on \(\triangle AMO\), and noting \(AM=\frac{1}{2}AB\), we get \(\frac{1}{AB}=\frac{2}{\sin x}\). By a similar procedure, this time on \(\triangle ACO\), we see \(\frac{1}{AB}=\frac{2}{\sin 2x}\), and again on \(\triangle ADO\), we have \(\frac{1}{AD}=\frac{2}{\sin3x}\). Therefore we have the equality: 
        
        \begin{align*}
            \frac{1}{\sin x}&=\frac{1}{\sin2x}+\frac{1}{\sin3x}
        \end{align*}

        Simplifying this yields \(\sin x\sin\frac{x}{2}\sin\frac{7x}{2}=0\). However, note that \(x\ne\frac{k\pi}{2},\ \frac{k\pi}{3}\) for \(k\in\mathbb{Z}\), otherwise, we have the issue of dividing by 0. Hence it must be the case that \(\sin\frac{7x}{2}=0\Rightarrow x=\frac{\pi}{7}+\frac{2k\pi}{7}\). Clearly the \(n\)-gon is not a square, so trivially it must be the case that \(x=\frac{\pi}{7}\). Therefore, the polygon must have \fbox{7} sides.
        \begin{figure}[h!]
            \centering
            \scalebox{0.8}{\begin{tikzpicture}[line cap=round,line join=round,>=triangle 45,x=1.0cm,y=1.0cm]
            \clip(-5.5,-5.5) rectangle (5.5,5.5);
            \draw [shift={(0.,0.)},line width=0.4pt] (0,0) -- (-67.5:0.3710168360883199) arc (-67.5:-22.5:0.3710168360883199) -- cycle;
            \draw [shift={(0.,0.)},line width=0.4pt] (0,0) -- (-22.5:0.3710168360883199) arc (-22.5:22.5:0.3710168360883199) -- cycle;
            \draw [shift={(0.,0.)},line width=0.4pt] (0,0) -- (22.5:0.3710168360883199) arc (22.5:67.5:0.3710168360883199) -- cycle;
            \draw[line width=0.4pt] (3.0808989941467813,-3.0808989941467813) -- (2.895390576102621,-3.2664074121909414) -- (3.080898994146781,-3.4519158302351016) -- (3.266407412190941,-3.2664074121909414) -- cycle; 
            \draw [line width=0.4pt,] (1.9134171618254487,-4.619397662556435)-- (4.619397662556434,-1.9134171618254485);
            \draw [line width=0.4pt,dash pattern=on 5pt off 5pt] (4.619397662556434,-1.9134171618254485)-- (4.619397662556434,1.9134171618254485);
            \draw [line width=0.4pt,dash pattern=on 5pt off 5pt] (4.619397662556434,1.9134171618254485)-- (1.913417161825449,4.619397662556434);
            \draw [line width=0.4pt,dash pattern=on 5pt off 5pt] (1.913417161825449,4.619397662556434)-- (-1.9134171618254483,4.619397662556434);
            \draw [line width=0.4pt] (1.9134171618254487,-4.619397662556435)-- (4.619397662556434,1.9134171618254485);
            \draw [line width=0.4pt] (1.9134171618254487,-4.619397662556435)-- (1.913417161825449,4.619397662556434);
            \draw [line width=0.4pt] (0.,0.)-- (1.9134171618254487,-4.619397662556435);
            \draw [line width=0.4pt] (0.,0.)-- (4.619397662556434,-1.9134171618254485);
            \draw [line width=0.4pt] (0.,0.)-- (4.619397662556434,1.9134171618254485);
            \draw [line width=0.4pt] (0.,0.)-- (1.913417161825449,4.619397662556434);
            \draw [line width=0.4pt] (0.,0.)-- (3.266407412190941,-3.2664074121909414);
            \begin{scriptsize}
            \draw[color=black] (-0.25016257386958146,3.362996814193775E-4) node {$O$};
            \draw[color=black] (4.98117481497573,2.078030581776008) node {$C$};
            \draw[color=black] (1.8275317082250102,-4.934187620293228) node {$A$};
            \draw[color=black] (5.111030707606642,-2.114459666022001) node {$B$};
            \draw[color=black] (2.050141809878002,4.916309377851651) node {$D$};
            \draw[color=black] (-1.9753908616802691,4.9719619032649) node {$E$};
            \draw[color=black] (3.4971074706224496,-3.394467750526703) node {$M$};
            \end{scriptsize}
            \end{tikzpicture}}
            \caption*{A regular $n$-gon}
        \end{figure}
    \end{solution}
    
    \begin{solution}[Write up by \Paiya]\hfil\medskip
    
      Let $d_k$ represent the diagonal from a point to the kth vertex adjacent to it.     For example, $d_1$ is a side of the polygon, $d_2$ is $\overline{AC}, d_3$ is $\overline{AD}$ and note that $d_k = d_{n - k}$ where n is the number of sides of the polygon. Reassign $D$ to be the vertex three vertices away from $A$ but on the opposite side of $B$ and $C;$ in other words, reflect $D$ over $\overline{OA}.$ Then, $AB = BC = d_1, AC = d_2, AD = d_3, BD = d_4,$ and $CD = d_5.$ By Ptolemy's Theorem, we get $$AB \cdot CD + BC \cdot AD = AC \cdot BD \iff d_1\left(d_5 + d_3\right) = d_2d_4.$$ Rearrange our given equation to get $$\dfrac{1}{d_1} = \dfrac{1}{d_2} + \dfrac{1}{d_3} \iff d_1\left(d_2 + d_3\right) = d_2d_3.$$ For both of these equations to be true, we can have $d_5 + d_3 = d_2 + d_3 \iff d_5 = d_2$ and $d_3 = d_4;$ this is true if \(n = \boxed{7}\).
    \end{solution}

\newpage 

\section{2p}

\SSbreak\\
\emph{Source: China Western MO 2003, Day 1 P2}\\
\emph{Proposer: .19\#9839 (434767660182405131)}\\
\emph{Problem ID: 51}\\
\emph{Date: 2020-11-20}\\
\SSbreak

\SSpsetQ{
    Let \(a_1,a_2,\ldots,a_{24n}\) be real numbers with \(\displaystyle\sum_{i=1}^{24n-1}(a_{i+1}-a_i)^2=1\).\medskip

    For some \(n>0\) the maximum value of \((a_{12n+1}+a_{12n+2}+\cdots+a_{24n})-(a_1+a_2+\cdots+a_{12n})\) is twice that of a prime.\bigskip

    What is the sum of the value of that prime and the corresponding value of \(n\)?
}\bigskip

\begin{solution}\hfil\medskip

    If we substitute \(x_{i+1}=a_{i+1}-a_i\), we get \(a_i=\sum_1^i x_r\), and thus our constraint becomes 
    \begin{equation*}
        \sum_{i=1}^{24n-1}(a_{i+1}-a_i)^2=\sum_{i=1}^{24n}x^2_i=1
    \end{equation*}
    
    Putting the bit we wish to be maximising in terms of the substitution gives:

    \begin{align*}
        \sum_{i=1}^{12n}a_i&=12nx_1+(12n-1)x_2+\cdots+2x_{12n-1}+x_{12n}\\
        \sum_{i=12n+1}^{24n}a_i&=12n(x_1+\cdots+x_{12n+1})+(12n-1)x_{12n+2}+\cdots 2x_{24n-1}+x_{24n}
    \end{align*}

    Hence, 
    \begin{equation*}
        \sum_{i=12n+1}^{24n}a_i-\sum_{i=1}^{12n}a_i=x_2+\cdots+(12n-1)x_{12n}+12nx_{12n+1}+(12n-1)x_{12n+2}+\cdots+x_{24n}
    \end{equation*}
    Then by Cauchy-Schwarz: 

    \begin{align*}
        \sum_{i=12n+1}^{24n}a_i-\sum_{i=1}^{12n}a_i&\leq \sqrt{\left((12n)^2+2\sum_{i=1}^{12n-1}i^2\right)\left(\sum_{i=1}^{24n}x_i^2\right)}=\sqrt{(12n)^2+\frac{12n(12n-1)(24n-1)}{3}}\\
        &\leq\sqrt{4n(2(12n)^2+1)}=2p
    \end{align*}

    Now we want values of \(n\) such that \(n(288n^2+1)=p^2\) for a prime \(p\). Since trivially for all \(n,\ n<288n^2+1\), we have \(n=1\) and \(288n^2+1=p^2\), hence \(p=17\), so the answer is \fbox{18}.
    \end{solution}

\section{Slippery Rooks}

    \SSbreak\\
    \emph{Source: AMOC 2019 December School Prep Problems C5}\\
    \emph{Proposer: ChristopherPi\#8528 (696497464621924394)}\\
    \emph{Problem ID: 54}\\
    \emph{Date: 2020-11-22}\\
    \SSbreak
    
    \SSpsetQ{
        MODSbot is trying to get rich by scamming MODS members out of their money, so it's devised a chess game on a \(2020\times2020\) chessboard for unsuspecting people to attempt before they can enter \textbf{.19's EPIC QoTD Party}. Suppose Brainy, Ishan, Nyxto, Adam, Bubble, Sharky and Christopher all get scammed by MODSbot, that is, MODSbot plays the chess game against all 7 at the same time on different boards.\bigskip

        The group decide to pool together their money which comes to a total of \(4.20\) BTC, and to play, they'll need to buy \(n\) batches of slippery rooks from MODSbot. A batch of slippery rooks contains one white and four black rooks, and each batch is sold at a price equivalent to \(0.069\) BTC per rook. Once the batches of rooks have been bought, the group may choose to distribute them in a way which allows all members to beat the game.\bigskip
       
        In the game, only one white rook may be placed on the board, and we define how slippery rooks move as follows: it slips along the row or column it's moved along and comes to rest on an empty square because it is obstructed by either the edge of the board or another rook. Initially, MODSbot places the rooks on the board randomly, and marks a square red. Then the person being scammed can choose any rook on each turn and move as allowed, and attempt to place the white rook on the red square in a finite number of moves.\bigskip
       
        The amount of money they  have left over after buying the smallest \(n\) batches rooks to guarantee that they all succeed in beating MODSbots game is \(k\) BTC. What is the value of \(1000k\)?}

    \begin{solution} [Write up by ChristopherPi]\hfil\medskip

        Consider simply the case of one person. We prove that three rooks are required, one white and two black.\\
        
        First we show that two are not enough: simply place the two rooks at corners of the board and mark any square not on the side of the board. It’s clear that neither rook can ever move to a square not on the side of the board. 
        Now we show three are enough.\\
        
        Suppose square \((a, b)\) is marked, where \((1, 1)\) is the bottom left corner and \((2020, 2020)\) is the top right corner.
        Trivially one can move the black rooks to \((1, 1)\) and \((2, 1)\) and the white rook to \((2020, 2020)\). 
        Next, simply “loop” the black rooks as follows: take the rook further left, and move it up, right, down and left such that it moves to the right of the rook originally on its right, and repeat until you place a black rook at \((a - 1, 1)\). Now if \(b\) is odd, move the white rook to \((a, 1)\) and the black rook at \((a - 2, 1)\) to \((2020, 2020)\); if it’s even, loop the leftmost black rook one more time to place it at \((a, 1)\).\\
        
        Now move the rook at \((a - 1, 1)\) to \((a - 1, 2020)\), and move the rook at \((2020, 2020)\) left then down to \((a, 2)\). Next we describe another “looping” procedure: take the rook with first coordinate a and smaller second coordinate, and move it right, up, left and down, so it now has first coordinate a and second coordinate larger than the other rook with first coordinate a. Repeat this until you place a rook at \((a, b)\) - since the colour of the rook placed at \((a, 1)\) is dependent on the parity of b, this ensures that the rook placed at \((a, b)\) must be a white rook.\\
        
        This procedure won’t work if either of \((a, b)\) is 1 or 2020, or both of \(a\) and \(b\) are either 2 or 2019. 
        In the first case, rotate the board such that \(a = 1\). Now place a black rook at \((2020, 2020)\). If \(b\) is odd, place the white rook at \((1, 1)\) and the other black rook at \((2, 1)\); else place the other black rook at \((1, 1)\) and the white rook at \((2, 1)\). Now use the first looping procedure until the white rook is placed at \((1, b)\) as required - since the position of the white rook depends on the parity of \(b\) this is certain to work. In the second case, rotate the board such that \((a, b) = (2, 2)\). Now you can trivially move the white rook to \((1, 1)\) and the black rooks to \((2, 1)\) and \((1, 2020)\). Now move the white rook up, right, up, left and down to place it at \((2, 2)\) as required.\\
        
        This shows that three is sufficient for one person. Hence the group must buy 7 batches because each of them needs a white rook, and one batch contains one white rook. Therefore, the answer is \(1000(4.2-7\cdot5\cdot0.069)=\fbox{1785}\).
    \end{solution}

\newpage

\section{Sets of Integer Solutions}

    \SSbreak \\
    \emph{China Mathematical Olympiad, 2005 Day 2 P6}\\
    \emph{Proposer: .19\#9839 (434767660182405131)}\\
    \emph{Problem ID: 55}\\
    \emph{Date: 2020-11-23}\\
    \SSbreak
       
     \SSpsetQ{ Define functions \(f\) and \(g\) such that \(f(a,b)=2^a3^b\), and \(g(c,d)=5^c7^d\), for \(a,b,c,d\in\Z_{\geq0}\).

    Given \(f(a,b)=1+g(c,d)\), what is the sum of all valid \(b\)'s, \(c\)'s and \(d\)'s, multiplied by the sum of all valid \(a\)'s?
        
    For example if we had valid solutions of \((a,b,c,d)=(1,1,2,4),(5,1,6,2),(0,0,0,0)\)

    Then the answer would be \((\underbrace{1+1+0}_{b's}+\underbrace{2+6+0}_{c's}+\underbrace{4+2+0}_{d's})\times(\underbrace{1+5+0}_{a's})=96\)}\bigskip
       
     \begin{solution}\hfil\medskip 

         We proceed by considering parity, for \(2^a3^b=5^c7^d+1\), we have the RHS as even, thus we must have \(a\geq 1\). 
         If we let \(b=0\), then for \(2^a-5^c\cdot7^d=1\), we have \(2^a\equiv 1\mod{5}\) for \(c\ne0\). 
         This gives \(a\equiv0\mod{4}\), so \(2^a-1\equiv0\mod{3}\). 
         But this clearly cannot be the case so we must have \(c=0\) when \(b=0\). 

         Hence, we consider \(2^a-7^d=1\). 
         Bashing gives \((a,d)=(1,0),\ (3,1)\). 
         These are the only such solutions as for \(a>4,\ 7^d\equiv -1\mod{16}\), but this is impossible. 
         So for the case of \(b=0\) all possible non-negative integer solutions are \((1,0,0,0),\ (3,0,0,1)\).\\

         Now let \(b>0\) and \(a=1\), so we now consider \(2\cdot3^b-5^c\cdot7^d=1\) under\(\mod{3}\), which gives \(-5^c7^d\mod{3}\). 
         Since \(7^d\equiv 1\mod{3}\), for all \(d\geq0\), we are left with \((-1)^c5^c\equiv 1\mod{3}\).
         Now \(5^c=\{1,2\}\mod{3}\), thus we see that we must have \(c\) being odd. Under\(\mod{5}\), we see that \(2\cdot3^b\equiv 1\mod{5}\), \(3^{b-1}\equiv 1\mod 5\). 
         As we observe that \(3^b\equiv\{3,4,2,1\}\mod{5}\), we must have \(b\equiv 1\mod{4}\). 
         If \(d\ne0\), then \(2\cdot3^b\equiv 1\mod{7}\). Again observe that \(3^b\equiv\{3,2,6,4,5,1\}\mod{7}\), we see \(b\equiv 4\mod{6}\). 
         But \(b\equiv 1\mod{4}\), so a contradiction arises, and thus \(d=0\) and hence \(2\cdot3^b-5^c=1\). 
         For \(b=1\), clearly \(c=1\). 
         So if \(b\geq 2\), then \(5^c\equiv-1\mod{9}\Rightarrow c\equiv 3\mod{6}\). 
         Therefore \(5^c+1\equiv0\mod{(5^3+1)}\Rightarrow 5^c+1\equiv0\mod{7}\), but this contradicts the fact that \(5^c+1=2\cdot3^b\). 
         Hence in this case we only have one solution \((a,b,c,d)=(1,1,1,0)\).\\

         Finally, consider the case where \(b>0\), and \(a\geq 0\). 
         Then we have \(5^c7^d\equiv-1\mod{4}\), and \(5^c7^d\equiv-1\mod{3}\), i.e. \((-1)^d\equiv -1\mod{4}\) and \((-1)^c\equiv -1 \mod{3}\). 
         Therefore we have that both \(c\) and \(d\) being odd. Thus, \(2^a3^b=5^c7^d+1\equiv4\mod 8\). 
         So \(a=2\) and thus \(4\cdot 3^b\equiv 1\mod{5}\) and \(4\cdot3^b\equiv 1\mod{7}\). 
         This gives \(b\equiv2\mod{12}\). 
         Substituting \(b=12k+2\) for \(k\in\Z_{\geq0}\), then \(5^c7^d=(2\cdot3^{6k+1}-1)(2\cdot3^{6k+1}+1)\).

         Now as \(\mathrm{gcd}(2\cdot3^{6k+1}+1,2\cdot3^{6k+1}-1)\), \(2\cdot3^{6k+1}-1\equiv0\mod{5}\), therefore \(2\cdot3^{6k+1}-1=5^a\) and \(2\cdot3^{6k+1}=7^d\). 
         If \(k\geq1\), then \(5^c\equiv-1\mod{9}\). 
         But this is impossible, so if \(k=0\), then \(b=2,\ c=1\), and \(d=1\).  
         Thus in this case, we have only one solution: \((a,b,c,d)=(2,2,1,1)\).

         Hence we can conclude all non-negative integer solutions are

         \begin{equation*}
            (a,b,c,d)=\begin{cases}
                 &(1,0,0,0)\\
                 &(3,0,0,1)\\
                 &(1,1,1,0)\\
                 &(2,2,1,1)
             \end{cases}
         \end{equation*}
        
        This then gives us an answer of \(\underbrace{(0+0+0+0+0+1+1+1+0+2+1+1)}_{7}\times\underbrace{(1+3+1+2)}_{7}=\fbox{49}\) 
     \end{solution}


\section{A Game of Deductions}

    \SSbreak\\   
    \emph{Source: Mathematics Admissions Test, 2014 Q6}\\
    \emph{Proposer: .19\#9839 (434767660182405131)}\\
    \emph{Problem ID: 56}\\
    \emph{Date: 2020-11-24}\\
    \SSbreak

     \SSpsetQ{
        CircleThm plays two rounds of a deduction game with Wen and Tan. In each round, CircleThm picks two integers \(x\) and \(y\) so that \(1\leq x\leq y\). He then whispers the sum of the two chosen integers to Wen, and the product of the two integers to Tan. Neither Wen nor Tan knows what CircleThm told the other. In the game, Tan and Wen must try to work out what the numbers \(x\) and \(y\) are using logical deductions.\medskip

        In the first round, suppose the product of the two chosen numbers, \(x_1\) and \(y_1\) is 8. \medskip

        Tan says \emph{"I don't know what \(x_1\) and \(y_1\) are"}
     
        Wen then says \emph{"I already knew that"}
     
        Tan then says \emph{"I now know \(x_1\) and \(y_1\)"}\medskip
     
        In the second round, suppose the sum of the two chosen numbers \(x_2\) and \(y_2\) is 5.\medskip
     
        Tan says \emph{"I don't know what \(x_2\) and \(y_2\) are"}
     
        Wen then says \emph{"I don't know what \(x_2\) and \(y_2\) are"}
     
        Tan then says \emph{"I don't know what \(x_2\) and \(y_2\) are"}
     
        Wen then says "\emph{I now know what \(x_2\) and \(y_2\) are"}\medskip
     
        What is \((x_1x_2+y_1y_2)^3\)?}\bigskip

    \begin{solution}\hfil\medskip 
     
        The first thing to observe is that Tan can only immediately deduce the values of \(\{x,y\}\) if, and only if, the prime factorisation of that number is unique - i.e. \(xy\) is prime.\\
        If the product of $\{x_1,y_1\}$ is 8, then the decomposition can be either $\{1,8\}$ or $\{2,4\}$. 
        However, if the decomposition was $\{2,4\}$, then Wen would have a sum of 6, so from their point of view the decomposition could potentially have been $\{1,5\}$, in which case Wen would have known that Tan would have known the decomposition as well - as the only way to achieve a product of 5 is from $\{1,5\}$. 
        Therefore the decomposition must have been $\{1,8\}$.
         
         For the second part, the decomposition's allowed are $\{1,4\}$ and $\{2,3\}$. 
         Assume that it is $\{1,4\}$. 
         Then, Tan only knows the product is 4, which mean Tan believes the decomposition is either $\{1,4\}$ or $\{2,2\}$. 
         If the decomposition was indeed $\{2,2\}$, then Wen would know that the sum is also 4, and thus that Wen would think that Tan sees a composition of $\{1,3\}$ or $\{2,2\}$. 
         Tan's first statement would show Wen that the decomposition was not $\{1,3\}$ (as then Tan would instantly know the decomposition)- in which Wen should know that the decomposition is $\{2,2\}$. 
         By Wen's first statement Tan then should know by their second statement that the decomposition is $\{1,4\}$; by Tan saying in their second statement that they don't know what the decomposition is, Wen then knows it must be $\{2,3\}$. 
         Thus the solution is \((1\cdot2+8\cdot3)^3 =\fbox{17576}\)
     \end{solution}

\newpage 
    
\section{Maximising Exponents}

     \SSbreak\\
     \emph{Source: Sixth Term Examination Paper - III, 1996 Q4}\\
     \emph{Proposer: .19\#9839 (434767660182405131)}\\
     \emph{Problem ID: 41}\\
     \emph{Date: 2020-11-26}\\
     \SSbreak

     \SSpsetQ{
         Consider the positive integer \(N\), and let \(\mathcal{Q}(N)\) denote the maximised product of integers that sum to \(N\).\medskip

         What is the sum of the exponents of the prime factorisation of \(\mathcal{Q}(1262)\)?\medskip

         For example: \(\mathcal{Q}(6)=\cdot3^2\), and \(mathcal{Q}(4)=2^2\), in the respective cases the sum of the exponents is 2, so the answer you would submit is 2. 

     }\bigskip

     \begin{solution}\hfil\medskip

         Let us work in the general case by first constructing a methodology which maximises product while keeping the sum constant. Consider \(N=n_1+n_2+\cdots+n_k\), and \(P(N)=n_1n_2\cdots n_k\). For any \(n_i\geq 4\), clearly we can replace it with \((n_i-2)+2\), which keeps the sum constant and increases the product (since \(n_i\leq 2(n_i-2)\)). Hence W.O.L.G assume all \(n_i<4\). This means that we can maximise the product of integers that sum to \(N\) by arranging it into some combination of 2's and 3's. If \(N\equiv0\mod{3}\) trivially we set all \(n_i\)'s equal 3. So \(\mathcal{Q}(3k)=3^{\frac{N}{3}}\) for an integer \(k\). In the case of \(N\equiv 1\mod{3}\), consider\(\mathcal{Q}(3k+1)\). We have \(\frac{N}{3}\) 3's in \(n_i\), and then a 1, or \(\frac{N}{3}-1\) 3's, and then a \(2^2\). Clearly in the latter case, the product is maximised. Hence \(\mathcal{Q}(3k+1)=2^2\cdot3^{\frac{N-4}{3}}\). A similar train of thought yields \(\mathcal{Q}(3k+2)=2\cdot3^{\frac{N-2}{3}}\) for \(N\equiv 2\mod{3}\)\medskip
         
         Therefore, we have the following result:
         
         \begin{equation*}
             \mathcal{Q}(N)=
             \begin{cases}
                 3^{\frac{N}{3}}\ &\mathrm{if}\ N \equiv 0 \mod 3\\
                 2^2\cdot3^{\frac{N-4}{3}}\ &\mathrm{if}\ N \equiv 1 \mod 3\\
                 2\cdot3^{\frac{N-2}{3}}\ &\mathrm{if}\ N \equiv 2 \mod 3
             \end{cases}
         \end{equation*}
         
         Since \(1262\equiv 2\mod 3\), we have \(\mathcal{Q}(1262)=2\cdot3^{\frac{1262-2}{3}}\), hence the sum of the exponents is\(1+420\), so the answer is \fbox{421}.
     \end{solution}\bigskip


\end{document}