\SSbreak\\
\emph{Source: Mathematics Admissions Test, 2020 Q1.D}\\
\emph{Proposer: \Pss}\\
\emph{Problem ID: 65}\\
\emph{Date: 2020-12-01}\\
\SSbreak

\SSpsetQ{
    The largest value achived by \(3\cos^2x+2\cos x+1\) can be represented as \(\frac{m}{n}\) as a fraction in lowest terms. Find \(m+n\).
}\bigskip

\begin{solution}\hfil\medskip
    
    We proceed by using the identity \(\cos^2x=1-\sin^2x\):

    \begin{align*}
        3\cos^2x+2\sin x+1&=3(1-\sin^2x)+2\sin x+1\\
        &=4+2\sin x-3\sin^2x\\
    \end{align*}

    This is a quadratic in \(\sin x\), specifically it is a convex parabola. Completing the square gives:

    \begin{equation*}
        4+2\sin x-3\sin^2x=\frac{13}{3}-3\left(\sin x-\frac{1}{3} \right)^2
    \end{equation*}

    Since for all values of \(\sin x\ne\frac{1}{3}\), the function \(f(x)=\frac{13}{3}-3\left(\sin x-\frac{1}{3}\right)^2\) is clearly decresaing, we must have a maximum at \(\sin x=\frac{1}{3}\), giving a value of \(\frac{13}{3}\). So the answer is \fbox{16}.

\end{solution}\bigskip

